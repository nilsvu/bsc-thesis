\documentclass{scrreprt}


% General
\usepackage[english]{babel}
\usepackage{hyperref}

% Fonts
\usepackage{fontspec,xunicode}
\setmainfont{Cardo}
\setsansfont{Avenir Next}
\setmonofont[Scale=MatchLowercase]{Menlo}

% Symbols
\newcommand{\LambdaCDM}{$\Lambda$CDM }

% Images
\newcommand{\plt}[4][\textwidth]{
	\begin{figure}[ht]
		\centering
		\includegraphics[width=#1]{#2}
		\caption{#4}
		\label{#3}
	\end{figure}
}

% Bibliography
\usepackage{natbib}
\bibliographystyle{plainnat}

% Appendix
\usepackage[toc,page]{appendix}

\usepackage{amsmath,amssymb}
\usepackage{esdiff} % derivatives
\usepackage{commath} % math macros
\usepackage{bbm} % blackboard style symbols

% Symbols
\newcommand{\Rscal}{\mathcal{R}} % Ricci scalar
\newcommand{\Rtens}{R} % Ricci tensor
\newcommand{\Riemtens}{R} % Riemann tensor
\newcommand{\Gtens}{G} % Einstein tensor
\newcommand{\Ttens}{T} % Energy-momentum tensor
\newcommand{\gdet}{|g|} % determinant of g
\newcommand{\Newtconst}{\mathrm{G}_\mathrm{N}} % Newtonian constant
\newcommand{\cosmconst}{\Lambda} % determinant of g
\newcommand{\hcross}{h_\times} % cross-polarized tensor perturbation
\newcommand{\hplus}{h_+} % plus-polarized tensor perturbation
\newcommand{\conft}{\eta} % conformal time
\newcommand{\efold}{\log a} % e-foldings
\newcommand{\Hcosm}{\mathrm{H}} % Hubble function
\newcommand{\Hconf}{\mathcal{H}} % conformal Hubble function
\newcommand{\eosp}{\omega} % equation of state parameter
\newcommand{\nexp}{n(\eosp)} % 3(1+w)
\newcommand{\spatcurv}{\kappa} % spatial curvature
\newcommand{\dens}{\rho} % density
\newcommand{\denscrit}{\rho_\textnormal{crit}} % density
\newcommand{\Dens}{\Omega} % normalized density
\newcommand{\symbdust}{\mathrm{d}} % normalized density
\newcommand{\symbrad}{\gamma} % normalized density
% Parametrization
\newcommand{\alphaM}{\alpha_\mathrm{M}} % parametrized friction
\newcommand{\cT}{c_\mathrm{T}} % parametrized propagation speed
\newcommand{\const}{\mathrm{const.}} % constant parameter

% Derivatives
\newcommand{\dt}[1]{\diff{#1}{t}}
\newcommand{\ddt}[1]{\diff[2]{#1}{t}}
\newcommand{\dconf}[1]{\dot{#1}}
\newcommand{\ddconf}[1]{\ddot{#1}}
\newcommand{\defold}[1]{#1^\prime}
\newcommand{\ddefold}[1]{#1^{\prime\prime}}

% Operators and other stuff
\newcommand{\vect}[1]{\boldsymbol{#1}}
\newcommand{\defeq}{:=}
\newcommand{\idmat}{\mathbbm{1}}
\newcommand{\covd}{\nabla}
\newcommand{\eul}{\mathrm{e}}



\title{Gravitational Waves in Modified Gravity}
% TODO: Subtitle about finding instability in bigravity
\author{Nils Fischer}
\date{July 28, 2015}

\begin{document}

\maketitle

\begin{abstract}
	By parametrization of the evolution equation of gravitational waves, I find constraints for the physical viability of a general modified gravity theory. In particular, I focus on a bimetric setting.
\end{abstract}


\tableofcontents


\chapter{Introduction}


\section{Gravitational Waves: Metric Pertubations in an FRW Universe} % Overview of Primordial Gravitational Waves

\begin{align}\label{eq:evolution}
	\ddconf{h} + 2 \Hconf \dconf{h} + c_T^2 k^2 h &= 0
\end{align}


\section{Dark Energy or Modified Gravity: Two Solutions for the Accelerating Expansion of our Universe} % Overview of the Dark Energy problem

- universe expands accelerated today, obtained by cosmological constant in Einstein-Hilbert action

\begin{align}\label{eq:einstein_hilbert_action}
	S[g] = \int \! \mathrm{d}^4 x \sqrt{-g} \left( R - 2\Lambda \right)
	\Rightarrow \; ... \ddt{a_\Lambda} > 0
\end{align}

- but needs to be \textbf{fine-tuned} and \textbf{why now?}

- either find model for $\Lambda$ (scalar field, ...) or \emph{modify gravity}


\section{Gravitational Waves in Modified Gravity}

- as shown in [], the evolution equation for gravitational waves takes the form []. In a theory of modified gravity, this equation will also be modified.

- For such a theory to be viable, the tensor pertubations must remain in limits set by observations.

- The evolution of gravitational waves can therefore provide constraints on physically viable modified gravity theories.

- To investigate the effects of various modifications of the evolution equation, I introduce parameters.



\chapter{Parametrization of Modified Gravitational Wave Evolution}


\section{Constant Friction}

I will first consider an additional friction term $\alpha_M$ in the evolution equation \citep{Pettorino2014}:

\begin{align}\label{eq:evolution_friction}
	\ddt{h} + \left( 3 + \alpha_M \right) \Hcosm \dt{h} + \frac{c_T^2 k^2}{a^2} h &= 0 \\
	\Leftrightarrow \quad \ddconf{h} + \left( 2 + \alpha_M \right) \Hconf \dconf{h} + c_T^2 k^2 h &= 0 \\
    \Leftrightarrow \quad \ddefold{h} + \left( \frac{\defold{\Hconf}}{\Hconf} + 2 + \alpha_M \right) \defold{h} + \frac{c_T^2 k^2}{\Hconf^2} h &= 0
\end{align}

\plt{plots/variation_alpha_m}{img:variation_alpha_m}{Increasing $\alpha_M$ introduces more friction, so the amplitude of the gravitational wave decreases more rapidly.}

For a constant $\alpha_M$, the differential equation can even be solved analytically in radiation or matter domination, where

\begin{align}
	&a(\conft) = a_0 \conft^n \quad \textrm{with} \quad n \defeq \frac{2}{1+3\eosp} \\
	\Rightarrow \quad &\Hconf = \frac{\dconf{a}}{a} = \frac{n}{\conft}
\end{align}

So \ref{eq:evolution_friction} becomes:

\begin{equation}\label{eq:evolution_friction_matter}
	\ddconf{h} + \left( 2 + \alpha_M \right) \frac{n}{\conft} \dconf{h} + c_T^2 k^2 h = 0
\end{equation}

The solution to \ref{eq:evolution_friction_matter} can be written in terms of Bessel functions:

\begin{equation}
	h(\conft) = \conft^{-p} \left[C_1 J_p(c_T k \conft) + C_2 Y_p(c_T k \conft) \right] \quad \textrm{with} \quad p = n(1+\frac{\alpha_M}{2})-\frac{1}{2}
\end{equation}

$J_p(x)$ and $Y_p(x)$ denote the Bessel functions of first and second kind, respectively. Their asymptotic behaviour for large $k$ or late times $\conft$ is:

\begin{align}
	J_p(x) \, &\rightarrow \, \sqrt{\frac{2}{\pi x}} \cos(x-\frac{2p+1}{4}\pi) + \mathcal{O}(x^{-\frac{3}{2}}) \\
	Y_p(x) \, &\rightarrow \, \sqrt{\frac{2}{\pi x}} \sin(x-\frac{2p+1}{4}\pi) + \mathcal{O}(x^{-\frac{3}{2}})
\end{align}

Therefore, we find the asymptotic behaviour for the tensor pertubations with friction $\alpha_M$ in radiation or matter domination as

\begin{align}
	h(\conft) &\propto \conft^{-p-\frac{1}{2}} = \conft^{-n(1+\frac{\alpha_M}{2})} \\
	\Rightarrow \: h(a) &\propto a^{\frac{-p-\frac{1}{2}}{n}} = a^{-(1+\frac{\alpha_M}{2})}
\end{align}

times fast oscillation.

This immediately yields a constraint for the additional friction $\alpha_M$ in this regime:

\begin{equation}
	\alpha_M \geq -2
\end{equation}


\subsection{Comparison to \cite{Amendola2015}}

In \citep{Amendola2015}, both the physical and reference metric tensor pertubations can be described as \ref{eq:evolution_friction_matter} with friction
\begin{align}
	\alpha_M^{g} &= 0 \\
	\alpha_M^{f} &= -3(1+\eosp) < -2
\end{align}

Clearly, the physical metric pertubations $h_g$ fall like $\frac{1}{a}$ as in \LambdaCDM, whereas the reference metric pertubations $h_f$ grow like $a^1$ in RDE and $a^\frac{1}{2}$ in MDE.


\section{Parametrized Friction}

More suitable to ... parametrization for the additional friction $\alpha_M$ could be time-dependent, e.g.

\begin{equation}
	\alpha_M = \alpha_{M0} a^\beta
\end{equation}

\plt{plots/variation_beta}{img:variation_beta}{For positive $\beta$, the evolution matches \LambdaCDM at early times and deviates at late times.}


\chapter{Gravitional Waves in Bigravity}

\chapter{Summary}


\begin{appendices}

\chapter{Mathematical Appendix}

\section{Bessel Functions}

\begin{align}
	f''(x) + \left(2p+1\right)\frac{1}{x} f'(x) + \left(\alpha^2 + \frac{\beta^2}{x^2}\right) f(x) = 0 \\
	\Rightarrow \: f(x) = x^{-p} \left[C_1 J_q(\alpha x) + C_2 Y_q(\alpha x)\right] \quad \textrm{with} \quad q = \sqrt{p^2 - \beta^2}
\end{align}

\subsection{Asymptotic Behaviour}

\end{appendices}


\bibliography{lit}


\end{document}
